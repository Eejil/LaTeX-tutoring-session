\documentclass{article}
\usepackage[utf8]{inputenc}

\usepackage[a4paper, total={165mm, 226mm}]{geometry}

\usepackage[english]{babel}
\usepackage{graphicx}
\usepackage{hyperref}

\usepackage{parskip}
\usepackage{fancyhdr}

\usepackage{mhchem} % https://docs.moodle.org/310/en/Chemistry_notation_using_mhchem

\title{\LaTeX{} Tutoring session}
\author{Author: Luciano Baldassi}
\date{April 2021}




\begin{document}

\maketitle

\tableofcontents


N.B. Check the source code as well as the PDF for examples and comments.
\section{Introduction}
\textbf{\textit{Welcome} to \LaTeX ! Let's get started !}


The \texttt{$\backslash$section$\{sectionName\}$}, \texttt{$\backslash$subsection$\{subsectionName\}$}, and \texttt{$\backslash$subsubsection$\{subsubsectionName\}$} allows to define section, subsection and subsubsection.


Comments are delimited by \%. Anything on the line after the \% symbol will not be compiled on the pdf.

Text can be break off on multiple line when writing in the .tex file. To break off text into different paragraph, just leave a blank line between the two block of text.


N.B. The symbols \&, \% or $\backslash$ have special uses in \LaTeX{} code. To write them, precede them with a backslash


\subsection{Commands and environments}
Commands are preceded by a backslash: $\backslash$

e.g. To write: \LaTeX write: % \texttt{
$\backslash$\texttt{LaTeX}
%}

Environment are delimited by \texttt{$\backslash$begin$\{environmentName\}$} and \texttt{$\backslash$end$\{environmentName\}$}

Lists, equations, table or figures are environments.

\section{Equations}

The equation environment looks like:
\begin{equation}
    \left( \frac{-\hbar}{2m} \frac{d^2}{dx^2} + V(x) \right) \cdot \psi(x) = E \cdot \psi(x)
    \label{eq:Schrodinger}
\end{equation}

To use MathMode, place your code into \$ ... \$ as such: $P=\frac{nRT}{V}$

\subsection{Chemical equations}

Using the package \texttt{mhchem} allows for easy formatting of chemical compounds and reactions using \texttt{$\backslash$ce\{...\}}

% $\backslash$ce\{...\}
\texttt{mhchem} works in equations, as well as both inside and outside Math Mode.

Example:
The following reaction shows the net reaction of the combustion of 1 glucose (\ce{C6H12O6}) into 6 \ce{CO2} and 6 \ce{H2O}:
\begin{equation}
    \ce{C6H12O6 + 6O2 -> 6CO2 + 6H2O}.
\end{equation}

Other reaction and arrows can be written that way, such as: 
$$\ce{H2(g) + I2(g) <=>[{k_1}][{k_2}] 2HI(g)}.$$
Refer to the documentation for the specifics
: \url{https://ftp.snt.utwente.nl/pub/software/tex/macros/latex/contrib/mhchem/mhchem.pdf}.

\section{In-document References}
To references Equations, Figures or Tables, place a \texttt{$\backslash$label$\{labelName\}$}
into your Equations, Figures or Tables and reference it using  \texttt{$\backslash$ref$\{labelName\}$}.

N.B. you need the package \texttt{hyperref}

\underline{Example:}

Eq.~\ref{eq:Schrodinger} display a well known equation (TISE).

Figure~\ref{fig:image1} shows an hilarious figure!

Table~\ref{tab:table1} shows an amazing table!




\section{Lists}

The \texttt{enumerate} environment create a numbered list:
\begin{enumerate}
    \item Element 1  
    \item Element 2
    \item Element 3
    \begin{itemize}
        \item Element 3.1
    \end{itemize}
\end{enumerate}

The \texttt{itemize} environment create a unnumbered list:
\begin{itemize}
    \item Element 1  
    \item Element 2
    \item Element 3
    \begin{itemize}
        \item Element 3.1
    \end{itemize}
\end{itemize}




\section{Floats}

\subsection{Figures}
To make figures, use the \texttt{figure} environment.

N.B. require the \texttt{graphicx} package.

\begin{figure}[ht] % the [ht] tells LaTeX to place the figure here (h) or on top of the page (t) if there is no space on the page.
    \centering
    \includegraphics[width=0.4\linewidth]{image1.jpg}
    % \includegraphics[width = 0.6\textwidth]{image1.jpg}
    \caption{This is an hilarious figure}
    \label{fig:image1}
\end{figure}


\subsection{Tables}
To make tables, use the \texttt{table} environment

\begin{table}[ht] 
    \centering
    \begin{tabular}{|l|c|c|r|} 
    \hline
        Column 1   & Column 2 & Column 3 & Column 4 \\ \hline
        Count:     &    22    &   35   &    43      \\ \hline
        Frequency: &  0.22\%  & 0.35\% &  0.43\%    \\ \hline
    \end{tabular}
    \caption{Such an amazing table !!}
    \label{tab:table1}
\end{table}


\section{Citations}
This document is an example of BibTeX using in bibliography management.
 To cite something, just use \texttt{$\backslash$cite$\{labelName\}$} as such \cite{DuBoff1967}.
Note that the reference are done automatically from the biblist.bib file in the overleaf project (Check the last lines of the source code on how to make the bibliography).

\section{Need anything}
There're a lot of templates on Overleaf. You don't need to start from scratch !


\textbf{In trouble ? Or want to do something more than is in this tutorial?}
\underline{Google it !} \LaTeX is a powerful tool; hopefully, there's a lot of resources on the internet.

\nocite{*}                  % This line just tell to print the whole content of the bib file. Comment to only display the references cited in text
\bibliographystyle{acm}
% \bibliographystyle{apalike} % This line define the citation style
\bibliography{biblist.bib}  % This line refers to the bib file in the project.



\end{document}
